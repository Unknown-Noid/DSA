\documentclass{ltjsarticle}

\usepackage{listings, xcolor}

\lstset{
    basicstyle = {\ttfamily}, % 基本的なフォントスタイル
    frame = {tbrl}, % 枠線の枠線。t: top, b: bottom, r: right, l: left
    breaklines = true, % 長い行の改行
    numbers = left, % 行番号の表示。left, right, none
    showspaces = false, % スペースの表示
    showstringspaces = false, % 文字列中のスペースの表示
    showtabs = false, % タブの表示
    keywordstyle = \color{blue}, % キーワードのスタイル。intやwhileなど
    commentstyle = {\color[HTML]{1AB91A}}, % コメントのスタイル
    identifierstyle = \color{black}, % 識別子のスタイル 関数名や変数名
    stringstyle = \color{brown}, % 文字列のスタイル
    captionpos = t % キャプションの位置 t: 上、b: 下
}
\renewcommand{\lstlistingname}{プログラム}

\begin{document}

\title{データ構造とアルゴリズム実験レポート\\
課題:最大公約数を求めるアルゴリズム}
\author{202110796 4クラス 高橋大粋}
\date{締切日:2024年10月16日\\
\today}
\maketitle

\section{基本課題}
この課題では、教科書リスト1.4(p.7)の「ユークリッドの互除法」に基づいたCプログラムgcd\_euclid.cを
作成し、作成したプログラムと教科書リスト1.1(p.3)の「最大公約数を求める素朴なアルゴリズム」に基づ
いたCプログラムgcd\_iter.cのリストおよび実行結果を示した。 
\subsection{gcd\_euclid.cの作成}
\subsubsection{実装の方針}\label{subsubsec:実装の方針}
まず、ユークリッドの互除法を実行するための機能をgcd\_euclid関数に、実行結果を表示するための機能をmain関数
に記述し、それぞれ実装した。main関数は別ファイルmain\_euclid.cに実装した。また、main関数は、int型データn,
mをコマンドライン引数で渡すことによって動作する。もし、引数が2つ以外ならエラーを出力する。gcd\_euclid関数
では入力されたn, mの大小関係を固定するために大きい方の数をnにする処理を実装してある。これにより、ユークリ
ッドの互除法を行いやすくなる。
\subsubsection{実装コードおよびコードの説明}
プログラム\ref{code:one}に、gcd\_euclid.cの主要部分を示す。\ref{subsubsec:実装の方針}で述べた、大きい方
の数をnにする処理は2~6行目の部分に相当する。\\
gcd\_euclid関数は整数型の引数を2つ取り、ユークリッドの互除法に基づき、nをmで割った余りがrのとき、nにmを代入、
mにrを代入してnをmで割り、この操作をmが0になるまで繰り返す。m=0のとき、nの値が最大公約数となるのでint型のnを
返す。mian関数では、コマンドラインで入力されたn, m, そしてgcd\_euclid関数で計算された最大公約数gcdの3つの値
が出力される。
\begin{lstlisting}[caption=gcd\_euclid.cの主要部, label=code:one, language=C]
int gcd_euclid(int n, int m) {
  if (n < m) {
      int tmp = m;
      m = n;
      n = tmp;
  }
  while (m != 0) {
      int r = n % m;
      n = m;
      m = r;
  }
  return n;
}
\end{lstlisting}
\end{document}