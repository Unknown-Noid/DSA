\documentclass{ltjsarticle}

\usepackage{listings, xcolor}

\lstset{
    basicstyle = {\ttfamily}, % 基本的なフォントスタイル
    frame = {tbrl}, % 枠線の枠線。t: top, b: bottom, r: right, l: left
    breaklines = true, % 長い行の改行
    numbers = left, % 行番号の表示。left, right, none
    showspaces = false, % スペースの表示
    showstringspaces = false, % 文字列中のスペースの表示
    showtabs = false, % タブの表示
    keywordstyle = \color{blue}, % キーワードのスタイル。intやwhileなど
    commentstyle = {\color[HTML]{1AB91A}}, % コメントのスタイル
    identifierstyle = \color{black}, % 識別子のスタイル 関数名や変数名
    stringstyle = \color{brown}, % 文字列のスタイル
    captionpos = b % キャプションの位置 t: 上、b: 下
    literate={_}{\_}1 % アンダースコアのエスケープ
}
\renewcommand{\lstlistingname}{プログラム}

\begin{document}

\title{データ構造とアルゴリズム実験レポート\\
課題:連結リスト,スタック,キュー}
\author{202110796 4クラス 高橋大粋}
\date{締切日:2024年10月28日\\
2024年10月28日}
\maketitle

\section{基本課題}
この課題では、教科書リスト2.2,2.3(p.24),2.4,2.5(p.25)の連結リストのCプログラムlinked\_list.c, 教科書リスト2.9,2.10
(p.32,33)のキューのCプログラムqueue.cのリストおよび実行結果を示した。
\subsection{連結リストの実装}
\subsubsection{実装の方針}\label{subsubsec:実装の方針1}
まず、連結リストの機能としてセルpの直後に新しいセルを挿入するinsert\_cell(), リストの先頭に新しいセルを挿入する
insert\_cell\_top(), セルpの直後のセルを削除するdelete\_cell(), リスト先頭のセルを削除するdelete\_cell\_top(), リストの要素を順に
標準出力するdisplay()をlinked\_list.cに実装した。また、main関数は別ファイルmain\_linked\_list.cに実装した。
\subsubsection{実装コードおよびコードの説明}\label{subsubsec:実装コードおよびコードの説明1}
プログラム\ref{code:one}に、linked\_list.cの主要部を示す。 \\ \indent
void insert\_cell(Cell *p, int d)関数は入力としてセルへのポインタpとint型のデータdを受け取り、新しいセルnew\_cellを作成し、
new\_cellのデータ部にdを、ポインタ部にpが指すポインタ部を代入する。そしてpが指すセルのポインタ部にnew\_cellのアドレス
を代入することでセルpの直後に新しいセルを挿入する。2行目ではCellサイズ分のメモリ領域を動的に確保し、そのアドレスを
ポインタnew\_cellに格納している。void insert\_cell\_top(int d)関数はinsert\_cell()関数と内容は殆ど同じだが、「pが指す
セルのポインタ部」を先頭セルのアドレスであるheadに置き換えることでリストの先頭に新しいセルを挿入する。
void delete\_cell(Cell *p)関数は入力としてセルへのポインタpを受け取り、セルへのポインタqにpが指すセルのポインタ部を
代入し、pが指すセルのポインタ部にqが指すセルのポインタ部を代入し、qが指すセルが使用している主記憶領域を開放することで、セルp
の直後のセルを削除する。void delete\_cell\_top(void)関数はセルへのポインタqに先頭セルへのポインタheadを代入して、headに
qが指すセルのポインタ部を代入し、qが指すセルが使用している主記憶領域を開放することでリストの先頭のセルを削除する。
void display(void)関数はセルへのポインタcurrentに先頭セルへのポインタheadを代入して、currentが指すデータ部をprintf()
関数でコマンドラインに出力する。そしてcurrentにcurrentが指すセルのポインタ部を代入することで直後のセルへ移動する。
これをcurrentがNULLになるまで繰り返し、改行する。
\begin{lstlisting}[caption=linked\_list.cの主要部, label=code:one, language=C, captionpos = b]
void insert_cell(Cell *p, int d) {
    Cell *new_cell = (Cell*)malloc(sizeof(Cell));
    new_cell->data = d;
    new_cell->next = p->next;
    p->next = new_cell;
}

void insert_cell_top(int d) {
  Cell *new_cell = (Cell*)malloc(sizeof(Cell));
  new_cell->data = d;
  new_cell->next = head;
  head = new_cell;
}

void delete_cell(Cell *p) {
    Cell *q = p->next;
    p->next = q->next;
    free(q);
}

void delete_cell_top(void) {
    Cell *q = head;
    head = q->next;
    free(q);
}

void display(void) {
    Cell *current = head;
    while(current != NULL){
        printf("%d ", current->data);
        current = current->next;
    }
    printf("\n");
}
\end{lstlisting}
\subsubsection{実行結果}\label{subsubsec:実行結果1}
まず、linked\_list.cを以下のmakeコマンドを実行してコンパイルし、プログラムを実行する。
\begin{lstlisting}
PS C:\Users\daiki\Desktop\DSA\kadai2> make
gcc   linked_list.o main_linked_list.o   -o linked_list
PS C:\Users\daiki\Desktop\DSA\kadai2> ./linked_list
1 2 3 
1 3
3  
\end{lstlisting}
main関数の主要部は以下の通りである。
\begin{lstlisting}
int main(void) {
    insert_cell_top(1);
    insert_cell(head, 3);
    insert_cell(head, 2);
    display();
    
    delete_cell(head);
    display();
    
    delete_cell_top();
    display();
    
    return EXIT_SUCCESS;
    }    
\end{lstlisting}
まず、insert\_cell\_top(1)でリストの先頭にデータ部が1の新しいセルを挿入した。insert\_cell(head, 3)
で先頭セルの直後にデータ部が3の新しいセルを挿入した。同様にデータ部が2の新しいセルを挿入し、
display()でリストの要素を標準出力した。
出力の1行目に1 2 3と表示されているため、セルをリストの先頭に挿入できること、セルを指定したセルの直後に挿入できること、という要件を
満たすことが確認できた。次に、delete\_cell(head)でheadの直後のセルを削除し、display()でリストの要素を標準出力
した。出力の2行目に1 3と表示されているため、先頭セルの直後の2が削除されていることが確認できた。そして、delete\_
cell\_top()でリストの先頭セルを削除し、display()でリストの要素を標準出力した。出力の3行目に3と表示されているため、
先頭セルの1が削除されていることが確認できた。以上の処理は\ref{subsubsec:実装コードおよびコードの説明1}で説明したように
動作している。よって、セルをリストの先頭に挿入できること、セルを指定したセルの直後に挿入できること、先頭セルを削除できること、
指定したセルの直後のセルを削除できること、という要件をすべて満たすことを確認できた。
\subsection{キューの実装}
\subsubsection{実装の方針}\label{subsubsec:実装の方針2}
まず、リングバッファのキューの機能としてサイズlenのキューを作成・初期化し、処理が完了したらQueue created!を標準出力し、
Queue型のポインタを返すcreate\_queue(), キューに整数を格納するenqueue(), キューから整数を取り出す
dequeue(), キューの要素を先頭から順に出力するdisplay(), キューを破棄し、処理が完了したらQueue deleted!を標準出力
するdelete\_queue()をqueue.cに実装した。また、main関数は別ファイルmain\_queue.cに実装した。
\subsubsection{実装コードおよびコードの説明}\label{subsubsec:実装コードおよびコードの説明2}
プログラム\ref{code:two}に、queue.cの主要部を示す。\\ \indent
Queue *create\_queue(int len)関数は引数にint型のlenを受け取り、キューqのlengthにlenを、frontとrearに-1を代入
して初期化する。そしてprintfでQueue created!を標準出力し、Queue型のポインタqを返す。void enqueue(Queue *q, int d)
関数は引数にQueue型のポインタqとint型のdを受け取り、キューが一杯であるかを判定し、一杯であるならば標準エラーを出力
する。そうでなければfrontが-1ならfrontに0を代入し、rearを+1する。このときリングバッファであるのでキュ
ーのサイズlengthで割った余りをrearに代入する。配列bufferのrearの指す位置に整数dを格納する。
int dequeue(Queue *q)関数は引数にQueue型のポインタqを取り、キューが空の時、標準エラーを出力し、そうでなければ
int型dataにfrontの指す位置のデータを代入し、frontとrearが同じ値なら両方-1を代入して初期化し、そうでなければリングバッファ
を考慮してfrontを+1し、dataの値を返す事により、キューから整数を取り出す。void display(Queue *q)は引数にQueue型の
ポインタqを取り、もしキューが空なら標準エラーを出し、そうでなければfrontからrearまでのキューの要素をprintfで
標準出力し、処理が終わったら改行する。void delete\_queue(Queue *q)関数は引数にQueue型のポインタqを取り、qがNULL
でなければqの配列とqの主記憶領域を開放し、printfでQueue deleted!を標準出力する。
\begin{lstlisting}[caption=queue.cの主要部, label=code:two, language=C,captionpos = b]
Queue *create_queue(int len) {
    Queue *q = (Queue *)malloc(sizeof(Queue));
    if (q == NULL) {
        exit(EXIT_FAILURE);
        return NULL;
    }
    q->buffer = (int *)malloc(len * sizeof(int));
    if (q->buffer == NULL) {
        exit(EXIT_FAILURE);
        free(q);
        return NULL;
    }
    q->length = len;
    q->front = -1;
    q->rear = -1;
    printf("Queue created!\n");
    return q;
}

void enqueue(Queue *q, int d) {
    if ((q->rear + 1) % q->length == q->front) {
        fprintf(stderr, "Queue is full! Cannot enqueue %d.\n", d);
        return;
    }
    if (q->front == -1) {
        q->front = 0;
    }
    q->rear = (q->rear + 1) % q->length;
    q->buffer[q->rear] = d;
}

int dequeue(Queue *q) {
    if (q->front == -1) {
        fprintf(stderr, "Queue is empty! Cannot dequeue.\n");
        return -1;
    }
    int data = q->buffer[q->front];
    if (q->front == q->rear) {
        q->front = -1;
        q->rear = -1;
    } else {
        q->front = (q->front + 1) % q->length;
    }
    return data;
}

void display(Queue *q) {
    if (q->front == -1) {
        fprintf(stderr, "Queue is empty! Nothing to display.\n");
        return;
    }
    int i = q->front;
    while (1) {
        printf("%d ", q->buffer[i]);
        if (i == q->rear) break;
        i = (i + 1) % q->length;
    }
    printf("\n");
}

void delete_queue(Queue *q) {
    if (q != NULL) {
        free(q->buffer);
        free(q);
        printf("Queue deleted!\n");
    }
} 
\end{lstlisting}
\subsubsection{実行結果}\label{subsubsec:実行結果2}
まず、queue.cを以下のmakeコマンドを実行してコンパイルし、プログラムを実行する。
\begin{lstlisting}
PS C:\Users\daiki\Desktop\DSA\kadai2> make queue
gcc    -c -o queue.o queue.c
gcc   queue.o main_queue.o   -o queue
PS C:\Users\daiki\Desktop\DSA\kadai2> ./queue
Queue created!
== Test 1: Enqueue one element and dequeue it ==
10
Dequeued: 10
Queue is empty! Nothing to display.

== Test 2: Enqueue multiple elements and dequeue them in order ==
20 30 40
Dequeued: 20
Dequeued: 30
40

== Test 3: Wrap-around case ==
40 50 60 70 80
Dequeued: 40
50 60 70 80
50 60 70 80 90

== Test 4: Queue empty and full conditions ==
Queue is full! Cannot enqueue 100.
Dequeued: 50
Dequeued: 60
Dequeued: 70
Dequeued: 80
Dequeued: 90
Queue is empty! Nothing to display.
Attempt to dequeue from empty queue:
Queue is empty! Cannot dequeue.
Queue deleted!  
\end{lstlisting}
main関数の主要部は以下の通りである
\begin{lstlisting}
int main(void) {
    Queue *test_q = create_queue(5);

    //1
    printf("== Test 1: Enqueue one element and dequeue it ==\n");
    enqueue(test_q, 10);
    display(test_q);
    printf("Dequeued: %d\n", dequeue(test_q));
    display(test_q);

    //2
    printf("\n== Test 2: Enqueue multiple elements and dequeue them in order ==\n");
    enqueue(test_q, 20);
    enqueue(test_q, 30);
    enqueue(test_q, 40);
    display(test_q);
    printf("Dequeued: %d\n", dequeue(test_q));
    printf("Dequeued: %d\n", dequeue(test_q));
    display(test_q);

    //3
    printf("\n== Test 3: Wrap-around case ==\n");
    enqueue(test_q, 50);
    enqueue(test_q, 60);
    enqueue(test_q, 70);
    enqueue(test_q, 80);  
    display(test_q);

    printf("Dequeued: %d\n", dequeue(test_q));  
    display(test_q);

    enqueue(test_q, 90);  
    display(test_q);

    //4
    printf("\n== Test 4: Queue empty and full conditions ==\n");

    enqueue(test_q, 100); 

    while (test_q->front != -1) {
        printf("Dequeued: %d\n", dequeue(test_q));
    }
    display(test_q);

    printf("Attempt to dequeue from empty queue:\n");
    dequeue(test_q);  

    delete_queue(test_q);

    return EXIT_SUCCESS;
}
\end{lstlisting}
Test1ではキューに整数を1つ格納し、それが取り出せていることを確認した。enqueue(q,10)とdisplay(q)で
キューに10を格納して表示し、dequeue(q)でキューから整数を取り出してprintf()で出力した。Test2では
enqueue()でキューに整数を複数格納し、display()で表示させた後に、dequeue()で格納した順に取り出して
printf()で出力した。Test3では配列の末尾と先頭にまたがる場合をテストした。enqueue()で5つ値を格納して
キューを一杯に満たした後、display()で表示させた。この状態でdequeue()で整数を取り出し、display()で
キューの要素を表示させると40が取り出され要素が4つになっていることを確認した。その後、enqueue()で再度
キューを満たし、display()で表示さ、要素が5つあることを確認した。Test4ではキューが空の時のdequeue()と
キューが満杯の時のenqueue()のエラーをテストした。現時点でキューが満杯なのでenqueue()するとキューが
満杯であり、enqueueできないというエラーが表示された。次にdequeue()でキューを空にした後でdequeue()を
すると、キューが空でdequeueできないというエラーが表示されることを確認した。
以上の処理は\ref{subsubsec:実装コードおよびコードの説明2}で説明したように動作している。
よって、課題で提示された要件をすべて満たすことを確認した。

\section{発展課題}
この課題では、ダミーのheadを用いる双方向循環リストの機能をdoublylinked\_list.cのCプログラムのリストおよび
実行結果を示した。
\subsection{双方向循環リストの実装}
\subsubsection{実装の方針}
まず、双方向循環リストの機能としてセルの生成および初期化を行い、Cell型のポインタを返す関数 CreateCell(), 
this\_cellの次にセルpを挿入する関数 InsertNext(Cell *this\_cell,Cell p), this\_cellの前にセルpを挿入する関数 
InsertPrev(Cell *this\_cell,Cell p), this\_cellをリストから削除する関数 DeleteCell(Cell *this\_cell), 
与えられた整数dを保持しているセルをthis\_cellから順に探し、見つかったらそのセルを返す。見つからなければ
NULLを返す関数 SearchCell(Cell *this\_cell,int d), リストの要素をthis\_cellから順に出力する関数 Display(Cell *this\_cell)
, 配列のデータを読み込みリストに追加する関数 ReadFromArray()
, 配列にリストの要素を書き出す関数 WriteToArray(), ファイルからデータを読み込みリストに追加する関数 ReadFromFile()
, ファイルにリストの要素を書き出す関数 WriteToFile()
をdoublylinked\_list.cに実装した。また、main関数は別ファイルmain\_doublylinked\_list.cに実装した。
\subsubsection{実装コードおよびコードの説明}\label{subsubsec:実装コードおよびコードの説明3}
プログラム\ref{code:three}に、doublylinked\_list.cの主要部を示す。\\ \indent
Cell *CreateCell(int d, bool is\_head)は引数にint型のd, 先頭かどうかを示すis\_headを取り、mallocでメモリを確保し、
確保に失敗した場合はプログラムを終了する。dataフィールドにdを代入し、is\_headフィールドにis\_headの値を代入して
先頭かどうかを指定する。セルが孤立状態になるようにprevとnextの両方のポインタを自身に設定する。is\_headがfalseの場合
は、リストに接続されていないことを示すエラーを表示し、最後に生成した新しいセルのポインタを返す。
void InsertNext(Cell *this\_cell, Cell *p)は引数としてthis\_cellと挿入するポインタpを取る。pのnextをthis\_cellのnext
に、prevをthis\_cellに設定し、this\_cellの次のセルのprevをpに、this\_cellのnextをpに更新することで、this\_cellの
次の位置にセルpを挿入する。
void InsertPrev(Cell *this\_cell, Cell *p)は引数にthis\_cellとpのポインタを取る。pのprevをthis\_cellのprev, 
nextをthis\_cellに設定し、this\_cellの前のセルのnextをpに、this\_cellのprevをpに更新することで、this\_cellの前にセルp
を挿入する。
void DeleteCell(Cell *this\_cell)は引数にthis\_cellを取る。this\_cellが削除対象であれば、その前後のセルをお互いに接続するよ
うに設定を変更し、this\_cellをリストから切り離す。最後にfreeでthis\_cellのメモリを解放する。
Cell *SearchCell(Cell *this\_cell, int d)はthis\_cellを起点にリストを巡回し、dataフィールドがdと等しいセルを見つけると、そ
のセルのポインタを返す。見つからなかった場合にはNULLを返す。this\_cellがNULLの場合もNULLを返す。
void Display(Cell *this\_cell)はthis\_cellがNULLの場合は処理を終了する。それ以外の場合、リストの各セルを巡回し、各セルのdataフィールドを標準出力に表示する。リストの終端に到達すると改行する。
void ReadFromArray(Cell *head, int arr[], int size)は引数としてヘッドセルheadと追加するデータを含む配列arr, 配列の要素数sizeを受け取る。配列内の各データに対し、新しいセルをCreateCellで作成し、InsertPrevを用いてヘッドセルの前に順に追加していく。
void WriteToArray(Cell *head, int arr[], int *size)はリストの各要素を配列に順次コピーし、配列の長さをsizeに代入する。
void ReadFromFile(Cell *head, const char *filename)は引数としてヘッドセルhead, 読み込み対象のファイル名filenameを取る。
ファイルを開き、整数データを読み込み、CreateCellとInsertPrevを使ってリストに追加する。読み込みが終わるとファイルを閉る。ファイルが開けなかった場合には標準エラー出力にエラーメッセージを出し、プログラムを終了する。
void WriteToFile(Cell *head, const char *filename)はファイルを開き、リスト内の各セルのデータを順にファイルに出力する。ファイルが開けない場合はエラーメッセージを出力し、プログラムを終了する。書き込みが完了したらファイルを閉じる。

\begin{lstlisting}[caption=doublylinked\_list.cの主要部, label=code:three, language=C,captionpos = b]
Cell *CreateCell(int d, bool is_head) {
    Cell *new_cell = (Cell *)malloc(sizeof(Cell));
    if (new_cell == NULL) {
        exit(EXIT_FAILURE);
    }
    new_cell->data = d;
    new_cell->is_head = is_head;
    new_cell->prev = new_cell;
    new_cell->next = new_cell;

    if (!is_head) {
        printf("Warning: Cell with data %d is not connected to the list.\n", d);
    }
    return new_cell;
}

void InsertNext(Cell *this_cell, Cell *p) {
    if (this_cell == NULL || p == NULL) return;
    p->next = this_cell->next;
    p->prev = this_cell;
    this_cell->next->prev = p;
    this_cell->next = p;
}

void InsertPrev(Cell *this_cell, Cell *p) {
    if (this_cell == NULL || p == NULL) return;
    p->prev = this_cell->prev;
    p->next = this_cell;
    this_cell->prev->next = p;
    this_cell->prev = p;
}

void DeleteCell(Cell *this_cell) {
    if (this_cell == NULL || this_cell->is_head) return;
    this_cell->prev->next = this_cell->next;
    this_cell->next->prev = this_cell->prev;
    free(this_cell);
}

Cell *SearchCell(Cell *this_cell, int d) {
    if (this_cell == NULL) return NULL;

    Cell *current = this_cell->next; 
    while (current != this_cell) {
        if (current->data == d) {
            return current; 
        }
        current = current->next;
    }
    return NULL; 
}

void Display(Cell *this_cell) {
    if (this_cell == NULL) return;
    Cell *current = this_cell->next;
    while (!current->is_head) {
        printf("%d ", current->data);
        current = current->next;
    }
    printf("\n");
}

void ReadFromArray(Cell *head, int arr[], int size) {
    for (int i = 0; i < size; i++) {
        InsertPrev(head, CreateCell(arr[i], false));
    }
}

void WriteToArray(Cell *head, int arr[], int *size) {
    int i = 0;
    Cell *current = head->next;
    while (!current->is_head) {
        arr[i++] = current->data;
        current = current->next;
    }
    *size = i;
}

void ReadFromFile(Cell *head, const char *filename) {
    FILE *file = fopen(filename, "r");
    if (file == NULL) {
        fprintf(stderr, "Failed to open file %s\n", filename);
        exit(EXIT_FAILURE);
    }
    int value;
    while (fscanf(file, "%d", &value) != EOF) {
        InsertPrev(head, CreateCell(value, false));
    }
    fclose(file);
}

void WriteToFile(Cell *head, const char *filename) {
    FILE *file = fopen(filename, "w");
    if (file == NULL) {
        fprintf(stderr, "Failed to open file %s\n", filename);
        exit(EXIT_FAILURE);
    }
    Cell *current = head->next;
    while (!current->is_head) {
        fprintf(file, "%d ", current->data);
        current = current->next;
    }
    fprintf(file, "\n");
    fclose(file);
}
\end{lstlisting}
\subsubsection{実行結果}\label{subsubsec:実行結果3}
まず、doublylinked\_list.cを以下のmakeコマンドを実行してコンパイルし、プログラムを実行する。
\begin{lstlisting}
PS C:\Users\daiki\Desktop\DSA\kadai2> make doublylinked_list
gcc    -c -o main_doublylinked_list.o main_doublylinked_list.c
gcc   doublylinked_list.o main_doublylinked_list.o   -o doublylinked_list
PS C:\Users\daiki\Desktop\DSA\kadai2> ./doublylinked_list   
Initial list created with head node.

== Inserting cells ==
Warning: Cell with data 1 is not connected to the list.
Warning: Cell with data 2 is not connected to the list.
Warning: Cell with data 3 is not connected to the list.
1 2 3

== Checking unconnected cell ==
Warning: Cell with data 4 is not connected to the list.
Warning: Cell with data 4 is not connected to the list.

== Deleting cells ==
2 3
2


== Reinserting and searching for cells ==
Warning: Cell with data 1 is not connected to the list.
Warning: Cell with data 2 is not connected to the list.
Warning: Cell with data 3 is not connected to the list.
1 2 3

Searching for cells:
Found cell with data 1 at address 000001f941cb1460
Found cell with data 2 at address 000001f941cb1480
Found cell with data 3 at address 000001f941cb14a0

List and all cells deleted.
\end{lstlisting}
main関数の主要部は以下の通りである
\begin{lstlisting}
int main(void) {
    Cell *head = CreateCell(0, true);
    printf("Initial list created with head node.\n");

    printf("\n== Inserting cells ==\n");
    Cell *first = CreateCell(1, false); 
    InsertNext(head, first);           

    Cell *middle = CreateCell(2, false); 
    InsertNext(first, middle);           

    Cell *last = CreateCell(3, false); 
    InsertPrev(head, last);           

    Display(head); 

    
    printf("\n== Checking unconnected cell ==\n");
    Cell *unconnected = CreateCell(4, false); 
    if (unconnected->next == unconnected && unconnected->prev == unconnected) {
        printf("Warning: Cell with data %d is not connected to the list.\n", unconnected->data);
    }

    printf("\n== Deleting cells ==\n");
    DeleteCell(first);  
    Display(head);       

    DeleteCell(last);    
    Display(head);      

    DeleteCell(middle);  
    Display(head);       

    printf("\n== Reinserting and searching for cells ==\n");
    Cell *new_first = CreateCell(1, false); 
    Cell *new_middle = CreateCell(2, false); 
    Cell *new_last = CreateCell(3, false);   

    InsertNext(head, new_first);    
    InsertNext(new_first, new_middle); 
    InsertPrev(head, new_last);    
    Display(head);                  

    printf("\nSearching for cells:\n");
    Cell *found;
    found = SearchCell(head, 1); 
    if (found != NULL) printf("Found cell with data 1 at address %p\n", (void *)found);
    else printf("Cell with data 1 not found.\n");

    found = SearchCell(head, 2);
    if (found != NULL) printf("Found cell with data 2 at address %p\n", (void *)found);
    else printf("Cell with data 2 not found.\n");

    found = SearchCell(head, 3); 
    if (found != NULL) printf("Found cell with data 3 at address %p\n", (void *)found);
    else printf("Cell with data 3 not found.\n");

    DeleteCell(new_first);
    DeleteCell(new_middle);
    DeleteCell(new_last);

    free(head);
    printf("\nList and all cells deleted.\n");

    return EXIT_SUCCESS;
}    
\end{lstlisting}
firstに先頭に挿入するセル、middleに中間に挿入するセル、lastに末尾に挿入するセルを割り当て、InsertNextとInsertPrev
で挿入する。この際、生成しただけで挿入していないセルがリストに繋がれていないことをメッセージとして出力している。Display
で表示させると1 2 3が表示された。次にリストに接続されていないセルを作成し、メッセージを表示させた。
Deleting cellsでは先頭セル、末尾セル、中間セルを削除できることを確認した。最後に新しく先頭、中間、末尾セルを作成し、
SearchCellにより先頭、中間、末尾セルを検索できることを確認した。\\ \indent
ReadFromArray, WriteToArray, WriteToFileの動作テストを以下のmain関数で行った。
\begin{lstlisting}
void main(void){
    Cell *head = CreateCell(0, true);
    printf("Initial list created with head node.\n");
   
    printf("\n== ReadFromArray() Test ==\n");
    int data[] = {1, 2, 3, 4, 5};
    int size = sizeof(data) / sizeof(data[0]);
    ReadFromArray(head, data, size);
    printf("List after reading from array: ");
    Display(head);

    printf("\n== WriteToArray() Test ==\n");
    int output[10]; 
    int output_size;
    WriteToArray(head, output, &output_size);
    printf("Array after writing from list: ");
    for (int i = 0; i < output_size; i++) {
        printf("%d ", output[i]);
    }
    printf("\n");

    printf("\n== WriteToFile() Test ==\n");
    const char *filename = "output.txt";
    WriteToFile(head, filename);
    printf("List data written to file '%s'\n", filename);

    printf("\n== ReadFromFile() Test ==\n");
  
    while (head->next != head) {
        DeleteCell(head->next);
    }
    printf("List cleared.\n");

    ReadFromFile(head, filename);
    printf("List after reading from file: ");
    Display(head);
    
    while (head->next != head) {
        DeleteCell(head->next);
    }
    free(head);
    printf("\nList and all cells deleted.\n");

    return EXIT_SUCCESS;  
} 
\end{lstlisting}
出力結果は以下の通り
\begin{lstlisting}
PS C:\Users\daiki\Desktop\DSA\kadai2> ./doublylinked_list   
Initial list created with head node.

== ReadFromArray() Test ==
Warning: Cell with data 1 is not connected to the list.
Warning: Cell with data 2 is not connected to the list.
Warning: Cell with data 3 is not connected to the list.
Warning: Cell with data 4 is not connected to the list.
Warning: Cell with data 5 is not connected to the list.
List after reading from array: 1 2 3 4 5

== WriteToArray() Test ==
Array after writing from list: 1 2 3 4 5

== WriteToFile() Test ==
List data written to file 'output.txt'

== ReadFromFile() Test ==
List cleared.
Warning: Cell with data 1 is not connected to the list.
Warning: Cell with data 2 is not connected to the list.
Warning: Cell with data 3 is not connected to the list.
Warning: Cell with data 4 is not connected to the list.
Warning: Cell with data 5 is not connected to the list.
List after reading from file: 1 2 3 4 5

List and all cells deleted.
\end{lstlisting}
例として配列{1,2,3,4,5}をReadFromArrayで読み込み、リストに追加してDsplayで表示させた。1 2 3 4 5と
表示されていることを確認した。WriteToArrayのテストではリストの要素を配列に書き出し、printfで配列の要素を
表示させた。1 2 3 4 5と表示されていることを確認した。WriteToFileのテストではリストのデータをoutput.txtに
書き出した。ReadFromFileのテストではリストをクリアしてから、ファイルからデータを読み込み、リストに追加し
てDsplayで表示させた。1 2 3 4 5と表示されていることを確認した。以上の処理は\ref{subsubsec:実装コードおよびコードの説明3}
で説明したように動作している。よって課題の要件をすべて満たしていることを確認した。
\end{document}